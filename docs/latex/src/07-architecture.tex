%------------------------
% Section 7: System Architecture
%------------------------
\section{ALS-Dimmer System Architecture}

Having covered the fundamental principles in Part~I, we now turn to the practical implementation of these concepts in the \textbf{ALS-Dimmer} system. ALS-Dimmer is a Linux-Daemon for adaptive brightness control designed for embedded Linux systems, with particular focus on automotive IVI (In-Vehicle Infotainment) applications.

\subsection{Design Philosophy}

ALS-Dimmer is built on three core principles:

\begin{enumerate}[leftmargin=*]
    \item \textbf{Hardware Agnostic:} Support diverse sensors and outputs through factory pattern with interface-based design
    \item \textbf{Configuration-Driven:} No code changes required for different hardware setups
    \item \textbf{Production-Ready:} Robust error handling, logging, systemd integration
\end{enumerate}

\subsection{High-Level Architecture}

Figure~\ref{fig:architecture} shows the main components and data flow:

\begin{figure}[ht]
\centering
\begin{tikzpicture}[
    scale=0.75,
    every node/.style={transform shape},
    node distance=1.5cm and 2cm,
    block/.style={rectangle, draw, fill=blue!10, text width=3.5cm, text centered, rounded corners, minimum height=1.2cm},
    interface/.style={rectangle, draw, fill=green!10, text width=3cm, text centered, rounded corners, minimum height=1cm},
    arrow/.style={->, >=stealth, thick}
]
    % Sensor layer
    \node[interface] (sensor) {Sensor Interface};
    \node[below=0.3cm of sensor, font=\footnotesize] (sensor_impl) {OPT4001, CAN ALS, File};

    % Zone mapper
    \node[block, right=of sensor] (zonemapper) {Zone Mapper};
    \node[below=0.2cm of zonemapper, font=\footnotesize, text width=3.5cm, align=center] {Selects zone based on\\ambient lux reading};

    % Brightness controller
    \node[block, right=of zonemapper] (controller) {Brightness Controller};
    \node[below=0.2cm of controller, font=\footnotesize, text width=4.0cm, align=center] {Calculates target,\\applies asymmetric transitions};

    % Output layer
    \node[interface, right=of controller] (output) {Output Interface};
    \node[below=0.3cm of output, font=\footnotesize] (output_impl) {DDC/CI, I2C, CAN, File};

    % JSON config (top)
    \node[block, above=1.5cm of zonemapper, fill=yellow!10] (config) {JSON Configuration};
    \node[below=0.2cm of config, font=\footnotesize, text width=4.0cm, align=center] {Defines sensors, outputs,\\zones, curves, timing};

    % Control interface (bottom)
    \node[block, below=1.5cm of controller, fill=orange!10] (control) {Control Interface};
    \node[below=0.2cm of control, font=\footnotesize, text width=3.5cm, align=center] {TCP/Unix socket\\for runtime control};

    % Arrows - main data flow
    \draw[arrow] (sensor) -- node[above, font=\footnotesize] {Lux} (zonemapper);
    \draw[arrow] (zonemapper) -- node[above, font=\footnotesize] {Zone} (controller);
    \draw[arrow] (controller) -- node[above, font=\footnotesize] {Brightness} (output);

    % Config arrows
    \draw[arrow, dashed] (config) -- (zonemapper);
    \draw[arrow, dashed] (config) -| (sensor);
    \draw[arrow, dashed] (config) -| (output);

    % Control arrows
    \draw[arrow, dashed, <->] (control) -- (controller);

\end{tikzpicture}
\caption{ALS-Dimmer High-Level Architecture}
\label{fig:architecture}
\end{figure}

\subsection{Core Components}

\subsubsection{Sensor Interface Layer}

Abstracts ambient light sensor hardware. Implementations include:
\begin{itemize}[leftmargin=*]
    \item \textbf{OPT4001:} TI OPT4001 high-precision I2C sensor (automotive-grade)
    \item \textbf{FPGA-OPT4001:} OPT4001 connected via FPGA I2C bridge
    \item \textbf{CAN ALS:} Sensor data via CAN bus (common in automotive)
    \item \textbf{File Sensor:} Reads lux from file (testing, simulation, integration with external systems)
\end{itemize}

All implementations conform to a common \texttt{SensorInterface} API, allowing hot-swapping via configuration.

\subsubsection{Zone Mapper}

Implements the zone-based mapping strategy described in Section~2.4. Responsibilities:
\begin{itemize}[leftmargin=*]
    \item Monitor current ambient lux reading
    \item Select appropriate zone based on lux boundaries
    \item Apply hysteresis to prevent oscillation
    \item Notify brightness controller of zone changes
\end{itemize}

Zones are fully configurable in JSON, including boundaries, curve types, and brightness ranges.

\subsubsection{Brightness Controller}

Core brightness calculation and transition control logic. Responsibilities:
\begin{itemize}[leftmargin=*]
    \item Calculate target brightness using zone's curve formula
    \item Implement asymmetric error-based step sizing (Section~2.6)
    \item Track current vs. target brightness
    \item Apply rate limiting and smoothing
    \item Handle manual overrides and modes
\end{itemize}

The controller runs in a control loop (typically 100--200 ms period) continuously updating brightness.

\subsubsection{Output Interface Layer}

Abstracts display brightness control hardware. Implementations include:
\begin{itemize}[leftmargin=*]
    \item \textbf{DDC/CI:} Control Display brightness via DDC/CI protocol (VESA standard)
    \item \textbf{I2C Dimmer:} Direct control of LED Driver ICs (e.g FPGA or TCON)
    \item \textbf{CAN Output:} Broadcast brightness via CAN bus to te CAN based Remote Display
    \item \textbf{File Output:} Write brightness to file (testing, integration with external systems)
\end{itemize}

Like sensors, all outputs conform to a common \texttt{OutputInterface} API.

\subsection{Operating Modes}

ALS-Dimmer supports three operating modes:

\begin{table}[h]
\centering
\caption{ALS-Dimmer Operating Modes}
\label{tab:operating_modes}
\small
\begin{tabular}{@{}llp{0.42\textwidth}@{}}
\toprule
\textbf{Mode} & \textbf{Trigger} & \textbf{Behavior} \\
\midrule
AUTO & Default startup & Fully automatic control based on ALS and zones \\
MANUAL & User override via control interface & Brightness fixed at user-specified value. ALS monitoring continues but brightness not updated. \\
MANUAL\_TEMPORARY & User adjustment in AUTO mode & Like MANUAL but reverts to AUTO after timeout (configurable, typically 30--60s). Allows temporary user adjustments without disabling ALS. \\
\bottomrule
\end{tabular}
\end{table}

\textbf{Rationale for MANUAL\_TEMPORARY:} Users may want to temporarily adjust brightness (e.g., reading a map in bright sun) without permanently disabling adaptive control. After a timeout, the system resumes automatic operation. \\
\textbf{Note:} MANUAL\_TEMPORARY mode is a read-only status flag, which reflects a temprory transition phase - it cannot be set through external tcp or unix-domain sockets.

\subsection{Daemon Architecture}

ALS-Dimmer runs as a Linux daemon with the following characteristics:

\begin{itemize}[leftmargin=*]
    \item \textbf{Single-threaded event loop:} Simplifies concurrency, reduces bugs
    \item \textbf{Systemd integration:} Can be managed via \texttt{systemctl} (start, stop, restart)
    \item \textbf{Structured logging:} Configurable log levels (trace, debug, info, warn, error)
    \item \textbf{Graceful shutdown:} Handles SIGINT/SIGTERM cleanly, saves state if needed
    \item \textbf{Hot configuration reload:} For Future extension (currently requires restart)
\end{itemize}

\subsection{Key Design Decisions}

\subsubsection{Why JSON Configuration?}

JSON was chosen over compiled configuration for several reasons:

\begin{itemize}[leftmargin=*]
    \item \textbf{No recompilation:} Hardware changes require only config file edits
    \item \textbf{Tooling:} JSON parsers widely available, easy to validate
    \item \textbf{Human-readable:} Engineers can understand and modify configs
    \item \textbf{Integration:} Easy to generate configs from build systems, provisioning tools
\end{itemize}

\subsubsection{Why Factory Pattern with Interfaces?}

The interface-based factory pattern provides:

\begin{itemize}[leftmargin=*]
    \item \textbf{Modularity:} New sensors/outputs added without touching core logic
    \item \textbf{Testing:} File-based sensors/outputs enable automated testing
    \item \textbf{Portability:} Same core daemon runs on different hardware platforms
    \item \textbf{Maintainability:} Clear separation of concerns
\end{itemize}

\subsubsection{Why Linux Daemon?}

Targeting embedded Linux (especially automotive IVI) drives the daemon architecture:

\begin{itemize}[leftmargin=*]
    \item \textbf{System service:} Runs continuously in background, starts at boot
    \item \textbf{Resource efficient:} Single-threaded, low CPU/memory footprint
    \item \textbf{Standard integration:} Works with systemd, journald, syslog
    \item \textbf{IPC-ready:} Unix Socket Control interface allows integration with HMI frameworks (Android, QNX, etc.)
\end{itemize}

\subsection{Summary}

ALS-Dimmer's architecture embodies the principles described in Part~I through a modular, configuration-driven design. Key strengths:

\begin{itemize}[leftmargin=*]
    \item \textbf{Hardware agnostic:} Supports diverse sensors and outputs
    \item \textbf{Configurable:} Zones, curves, timing all defined in JSON
    \item \textbf{Production-ready:} Systemd integration, robust error handling
    \item \textbf{Testable:} File-based interfaces enable simulation and CI/CD
\end{itemize}

The next sections detail the implementation of each component.
