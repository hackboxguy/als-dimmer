\section{Human Visual Perception}

\subsection{Vision Modes}

The human visual system operates in three distinct modes depending on ambient light levels:

\begin{description}
    \item[Photopic vision] ($>$3 cd/m², daylight conditions) -- Mediated by cone cells in the retina, enables color perception and high visual acuity. Optimal for reading, detail work, and color-critical tasks.

    \item[Scotopic vision] ($<$0.01 cd/m², night conditions) -- Mediated by rod cells, provides monochrome vision with high sensitivity to light but poor spatial resolution. Night-adapted eyes can detect very dim sources.
    
    \item[Mesopic vision] (0.01--3 cd/m², twilight) -- Transition range where both rods and cones are active. Most automotive nighttime driving occurs in this range.
\end{description}

For display brightness control, the key insight is that photopic vision (cone-mediated) dominates in all practical use cases except extreme darkness. However, \textbf{adaptation between these modes takes time}, which drives asymmetric transition requirements.

\subsection{Light Adaptation Mechanisms}

The human eye adapts to changing light levels through two primary mechanisms:

\textbf{1. Pupil Response (Fast)}
\begin{itemize}[noitemsep]
    \item \textbf{Constriction} (dark → bright): 100--500 ms
    \item \textbf{Dilation} (bright → dark): 2--5 seconds to noticeably widen, followed by slower chemical processes
    \item Provides ~16:1 dynamic range (2mm to 8mm diameter)
\end{itemize}

\textbf{2. Photochemical Adaptation (Slow)}
\begin{itemize}[noitemsep]
    \item Rhodopsin (rod photopigment) regeneration: 5--30 minutes
    \item Enables vision from starlight to sunlight (~1,000,000:1 range)
    \item Much slower than mechanical pupil response
\end{itemize}

\subsection{The Weber-Fechner Law}

A fundamental principle of human perception states that perceived brightness changes are proportional to the \textbf{logarithm} of the actual physical brightness change:

\begin{equation}
\text{Perceived Brightness} \propto \log(\text{Physical Brightness})
\label{eq:weber_fechner}
\end{equation}

\textbf{Practical implications:}
\begin{itemize}[noitemsep]
    \item Doubling brightness from 10\% to 20\% feels like the same increase as 50\% to 100\%
    \item Linear brightness adjustments feel "wrong" -- large steps at low end, imperceptible at high end
    \item Logarithmic mapping aligns with human perception for optimal user experience
\end{itemize}

\subsection{Asymmetric Adaptation Times}

The most critical insight for display control design:

\begin{center}
\textbf{Dark → Bright adaptation is FAST (pupil constriction)}\\
\textbf{Bright → Dark adaptation is SLOW (pupil dilation + chemistry)}
\end{center}

Table \ref{tab:adaptation_times} quantifies these differences:

\begin{table}[h]
\centering
\begin{tabular}{|l|c|p{0.45\textwidth}|}
\hline
\textbf{Transition} & \textbf{Time} & \textbf{Display Brightness Strategy} \\
\hline
Tunnel exit & 100--500 ms & \textbf{Fast ramp up} (1--2 sec) prevents temporary blindness \\
(Dark → Bright) & (pupil constriction) & Large step sizes, aggressive convergence \\
\hline
Tunnel entry & 2--30 min & \textbf{Slow ramp down} (3--6 sec) prevents glare \\
(Bright → Dark) & (full adaptation) & Small step sizes, gradual convergence \\
\hline
\end{tabular}
\caption{Human adaptation times and display control implications}
\label{tab:adaptation_times}
\end{table}

\subsection{Design Requirements}

These physiological constraints translate to specific engineering requirements:

\begin{enumerate}
    \item \textbf{Logarithmic mapping} in extreme ranges (night/outdoor) to match perception
    \item \textbf{Linear mapping} acceptable in moderate ranges (indoor) for simplicity
    \item \textbf{Asymmetric step sizing}: Large steps for brightening, small for dimming
    \item \textbf{Temporal smoothing}: Prevent rapid oscillations that cause eye strain
    \item \textbf{Minimum brightness}: Never fully black (safety, readability)
    \item \textbf{Maximum brightness}: Panel capability, but consider power/thermal limits
\end{enumerate}

The next section introduces \textbf{zone-based mapping} as the solution that addresses all these requirements simultaneously.
