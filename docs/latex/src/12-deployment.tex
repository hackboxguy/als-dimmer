%------------------------
% Section 12: Deployment and System Integration
%------------------------
\section{Building, Installing, and Deploying ALS-Dimmer}

This section covers building the daemon from source, installing it on target systems, and integrating it with systemd for production deployment.

\subsection{Build System: CMake}

ALS-Dimmer uses CMake for cross-platform builds.

\subsubsection{Dependencies}

\textbf{Required:}
\begin{itemize}[leftmargin=*]
    \item C++17 compiler (GCC 7+, Clang 5+)
    \item CMake 3.12 or later
    \item nlohmann/json (JSON parsing library)
    \item spdlog (logging library)
\end{itemize}

\textbf{Optional (hardware-specific):}
\begin{itemize}[leftmargin=*]
    \item libi2c-dev (for I2C sensors/outputs)
    \item libsocketcan (for CAN interfaces)
\end{itemize}

\subsubsection{Building from Source}

\begin{lstlisting}[language=bash, caption={Standard CMake Build}]
# Clone repository
git clone https://github.com/your-org/als-dimmer.git
cd als-dimmer

# Create build directory
mkdir build && cd build

# Configure build
cmake -DCMAKE_BUILD_TYPE=Release ..

# Build
make -j$(nproc)

# Optional: Run tests
make test
\end{lstlisting}

\textbf{Build output:} Executable at \texttt{build/als-dimmer}

\subsubsection{Cross-Compilation for Embedded Targets}

\textbf{Example: ARM64 Yocto-based Linux}

\begin{lstlisting}[language=bash, caption={Cross-Compilation with Yocto SDK}]
# Source Yocto SDK environment
source /opt/poky/3.1/environment-setup-aarch64-poky-linux

# Configure for cross-compilation
cmake -DCMAKE_BUILD_TYPE=Release \
      -DCMAKE_TOOLCHAIN_FILE=toolchain-yocto-arm64.cmake \
      ..

# Build
make -j$(nproc)
\end{lstlisting}

\textbf{Example: Custom ARM toolchain}

\begin{lstlisting}[language=bash, caption={Cross-Compilation with Custom Toolchain}]
cmake -DCMAKE_BUILD_TYPE=Release \
      -DCMAKE_C_COMPILER=arm-linux-gnueabihf-gcc \
      -DCMAKE_CXX_COMPILER=arm-linux-gnueabihf-g++ \
      ..
\end{lstlisting}

\subsubsection{CMake Build Options}

\begin{table}[h]
\centering
\caption{CMake Configuration Options}
\label{tab:cmake_options}
\begin{tabular}{@{}llp{0.4\textwidth}@{}}
\toprule
\textbf{Option} & \textbf{Default} & \textbf{Description} \\
\midrule
\texttt{CMAKE\_BUILD\_TYPE} & Release & Build type: Debug, Release, RelWithDebInfo \\
\texttt{ENABLE\_TESTING} & ON & Build unit tests \\
\texttt{ENABLE\_OPT4001} & ON & Include OPT4001 sensor support \\
\texttt{ENABLE\_CAN} & ON & Include CAN bus support \\
\texttt{ENABLE\_DDCCI} & ON & Include DDC/CI output support \\
\texttt{INSTALL\_SYSTEMD} & ON & Install systemd service file \\
\bottomrule
\end{tabular}
\end{table}

\textbf{Example: Minimal build without CAN}

\begin{lstlisting}[language=bash]
cmake -DCMAKE_BUILD_TYPE=Release \
      -DENABLE_CAN=OFF \
      ..
\end{lstlisting}

\subsection{Installation}

\subsubsection{System Installation}

\begin{lstlisting}[language=bash, caption={Install to System Directories}]
# Install binary, config, and systemd service
sudo make install

# Default installation paths:
# /usr/local/bin/als-dimmer            (executable)
# /etc/als-dimmer/config.json          (example config)
# /etc/systemd/system/als-dimmer.service  (systemd unit)
\end{lstlisting}

\subsubsection{Custom Installation Prefix}

\begin{lstlisting}[language=bash, caption={Install to Custom Location}]
cmake -DCMAKE_INSTALL_PREFIX=/opt/user ..
make
sudo make install

# Installs to:
# /opt/user/bin/als-dimmer
# /opt/user/etc/als-dimmer/config.json
\end{lstlisting}

\subsection{Systemd Service Integration}

\subsubsection{Service Unit File}

ALS-Dimmer includes a systemd service file for automatic startup:

\begin{lstlisting}[caption={/etc/systemd/system/als-dimmer.service}, basicstyle=\ttfamily\scriptsize]
[Unit]
Description=ALS-Dimmer Adaptive Brightness Control Daemon
After=network.target

[Service]
Type=simple
ExecStart=/usr/local/bin/als-dimmer \
          --config /etc/als-dimmer/config.json \
          --log-level info
Restart=on-failure
RestartSec=5
User=root
StandardOutput=journal
StandardError=journal

[Install]
WantedBy=multi-user.target
\end{lstlisting}

\subsubsection{Service Management Commands}

\begin{lstlisting}[language=bash, caption={Systemd Service Control}]
# Enable service (start at boot)
sudo systemctl enable als-dimmer

# Start service immediately
sudo systemctl start als-dimmer

# Check service status
sudo systemctl status als-dimmer

# View logs
sudo journalctl -u als-dimmer -f

# Restart after config changes
sudo systemctl restart als-dimmer

# Stop service
sudo systemctl stop als-dimmer

# Disable service (don't start at boot)
sudo systemctl disable als-dimmer
\end{lstlisting}

\subsection{Configuration Deployment}

\subsubsection{Config File Location}

The daemon expects configuration at \texttt{/etc/als-dimmer/config.json} by default, but this can be overridden via command-line argument:

\begin{lstlisting}[language=bash]
als-dimmer --config /path/to/custom/config.json
\end{lstlisting}

\subsubsection{Deploying Configurations}

\textbf{Approach 1: Manual deployment}

\begin{lstlisting}[language=bash]
sudo mkdir -p /etc/als-dimmer
sudo cp my-automotive-config.json /etc/als-dimmer/config.json
sudo systemctl restart als-dimmer
\end{lstlisting}

\textbf{Approach 2: Yocto recipe}

For Yocto-based builds, include config in recipe:

\begin{lstlisting}[caption={als-dimmer.bb (Yocto Recipe Snippet)}, basicstyle=\ttfamily\scriptsize]
do_install_append() {
    install -d ${D}${sysconfdir}/als-dimmer
    install -m 0644 ${WORKDIR}/automotive-config.json \
        ${D}${sysconfdir}/als-dimmer/config.json
}
\end{lstlisting}

\subsection{Runtime Control and Debugging}

\subsubsection{Command-Line Arguments}

\begin{table}[h]
\centering
\caption{Command-Line Arguments}
\label{tab:cli_args}
\begin{tabular}{@{}llp{0.42\textwidth}@{}}
\toprule
\textbf{Argument} & \textbf{Default} & \textbf{Description} \\
\midrule
\texttt{--config FILE} & (required) & Path to JSON configuration file \\
\texttt{--log-level LEVEL} & info & Log level: trace, debug, info, warn, error \\
\texttt{--foreground} & (off) & Run in foreground (don't daemonize) \\
\texttt{--help} & - & Show help message and exit \\
\texttt{--version} & - & Show version and exit \\
\bottomrule
\end{tabular}
\end{table}

\subsubsection{Debugging in Foreground}

For development and debugging, run in foreground with verbose logging:

\begin{lstlisting}[language=bash, caption={Run in Foreground with Debug Logging}]
./als-dimmer --config /etc/als-dimmer/config.json \
             --log-level debug \
             --foreground
\end{lstlisting}

This outputs logs directly to terminal, useful for troubleshooting sensor/output issues.

\subsubsection{Inspecting Logs}

\begin{lstlisting}[language=bash, caption={View Systemd Journal Logs}]
# View recent logs
sudo journalctl -u als-dimmer --since "10 minutes ago"

# Follow logs in real-time
sudo journalctl -u als-dimmer -f

# Filter by log level
sudo journalctl -u als-dimmer -p warning
\end{lstlisting}

\subsection{Android IVI Integration}

For Android-based IVI systems, ALS-Dimmer typically runs as a native daemon alongside the Android framework.

\subsubsection{Integration Steps}

\begin{enumerate}[leftmargin=*]
    \item \textbf{Build:} Cross-compile for Android target (ARM/ARM64)
    \item \textbf{Install:} Place binary in \texttt{/system/bin/} or \texttt{/vendor/bin/}
    \item \textbf{Init script:} Add entry to \texttt{init.rc} to start daemon at boot
    \item \textbf{SELinux policy:} Grant permissions for I2C, CAN, socket access
    \item \textbf{HMI integration:} Android app communicates via Unix socket
\end{enumerate}

\textbf{Example init.rc entry:}

\begin{lstlisting}[caption={init.rc Entry for Android}]
service als-dimmer /vendor/bin/als-dimmer \
    --config /vendor/etc/als-dimmer/config.json \
    --log-level info
    class main
    user root
    group system
    oneshot
\end{lstlisting}

\subsection{Validation and Testing}

\subsubsection{Manual Testing Procedure}

\begin{enumerate}[leftmargin=*]
    \item \textbf{Verify service is running:}
    \begin{lstlisting}[language=bash]
sudo systemctl status als-dimmer
    \end{lstlisting}

    \item \textbf{Query status via control interface:}
    \begin{lstlisting}[language=bash]
echo '{"version":"1.0","command":"get_status"}' | \
    nc -U /tmp/als-dimmer.sock
    \end{lstlisting}

    \item \textbf{Test manual brightness control:}
    \begin{lstlisting}[language=bash]
echo '{"version":"1.0","command":"set_brightness",
      "params":{"brightness":50}}' | \
    nc -U /tmp/als-dimmer.sock
    \end{lstlisting}

    \item \textbf{Monitor logs:}
    \begin{lstlisting}[language=bash]
sudo journalctl -u als-dimmer -f
    \end{lstlisting}

    \item \textbf{Simulate lux changes} (if using file sensor):
    \begin{lstlisting}[language=bash]
echo "100.0" > /tmp/ambient_lux.txt
echo "1000.0" > /tmp/ambient_lux.txt
    \end{lstlisting}

    \item \textbf{Verify brightness changes:}
    \begin{lstlisting}[language=bash]
watch -n 1 'echo "{\"version\":\"1.0\",
    \"command\":\"get_status\"}" | nc -U /tmp/als-dimmer.sock'
    \end{lstlisting}
\end{enumerate}

\subsubsection{Automated Testing}

Unit tests verify zone mapping, curve calculations, and transition logic:

\begin{lstlisting}[language=bash]
# Build and run tests
cmake -DENABLE_TESTING=ON ..
make
make test

# Alternatively, run with verbose output
ctest --verbose
\end{lstlisting}

\subsection{Troubleshooting Common Issues}

\begin{table}[h]
\centering
\caption{Common Deployment Issues}
\label{tab:troubleshooting}
\begin{tabular}{@{}p{0.28\textwidth}p{0.62\textwidth}@{}}
\toprule
\textbf{Issue} & \textbf{Solution} \\
\midrule
Service fails to start & Check logs (\texttt{journalctl -u als-dimmer}). Verify config file path and permissions. \\
Sensor initialization fails & Verify I2C/CAN device permissions. Run as root or add user to appropriate groups. \\
Brightness not changing & Check output device permissions. Verify sensor is operational (\texttt{get\_status} command). \\
High CPU usage & Increase \texttt{loop\_period\_ms} (longer interval) or raise \texttt{update\_interval\_ms} to reduce control-loop frequency; also check for sensor read errors. \\
Socket connection refused & Verify socket path. Check if daemon is running. Ensure Unix socket has correct permissions. \\
\bottomrule
\end{tabular}
\end{table}

\subsection{Summary}

Deploying ALS-Dimmer involves:

\begin{itemize}[leftmargin=*]
    \item \textbf{Building:} CMake-based build system, supports cross-compilation
    \item \textbf{Installing:} Standard Linux paths, systemd integration
    \item \textbf{Configuring:} JSON config deployed to \texttt{/etc/als-dimmer/}
    \item \textbf{Testing:} Manual testing via control interface, automated unit tests
    \item \textbf{Monitoring:} Systemd journal for logs, status queries for runtime state
\end{itemize}

The next section concludes the document with a summary and recommendations.
