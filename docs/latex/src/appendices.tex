%------------------------
% Appendices
%------------------------
\appendix

\section{Complete Configuration Examples}

This appendix provides complete, tested JSON configurations for common deployment scenarios.

\subsection{Appendix A.1: Automotive IVI - OPT4001 + DDC/CI}

\begin{lstlisting}[caption={Automotive Configuration (Full)}, basicstyle=\ttfamily\scriptsize]
{
    "sensor": {
        "type": "opt4001",
        "i2c_bus": "/dev/i2c-1",
        "i2c_address": "0x44"
    },
    "output": {
        "type": "ddcci",
        "i2c_bus": "/dev/i2c-5",
        "vcp_code": "0x10"
    },
    "zones": [
        {
            "name": "night",
            "min_lux": 0.0,
            "max_lux": 5.0,
            "curve_type": "linear",
            "min_brightness": 5,
            "max_brightness": 18,
            "up_gain": 0.28,
            "down_gain": 0.10
        },
        {
            "name": "indoor",
            "min_lux": 5.0,
            "max_lux": 1000.0,
            "curve_type": "logarithmic",
            "min_brightness": 18,
            "max_brightness": 65,
            "up_gain": 0.25,
            "down_gain": 0.12
        },
        {
            "name": "outdoor",
            "min_lux": 1000.0,
            "max_lux": 100000.0,
            "curve_type": "logarithmic",
            "min_brightness": 65,
            "max_brightness": 100,
            "up_gain": 0.35,
            "down_gain": 0.18
        }
    ],
    "control": {
        "loop_period_ms": 120,
        "error_threshold": 0.5,
        "min_step_size": 1.0,
        "hysteresis_percent": 10.0,
        "manual_timeout_seconds": 60
    },
    "control_interface": {
        "tcp": {
            "enabled": false
        },
        "unix": {
            "enabled": true,
            "socket_path": "/tmp/als-dimmer.sock"
        }
    }
}
\end{lstlisting}

\subsection{Appendix A.2: CAN-Based System}

\begin{lstlisting}[caption={CAN-Based Configuration (Full)}, basicstyle=\ttfamily\scriptsize]
{
    "sensor": {
        "type": "can_als",
        "can_interface": "can0",
        "can_id": "0x300",
        "data_offset": 2,
        "data_length": 2,
        "endianness": "big",
        "scale_factor": 0.1
    },
    "output": {
        "type": "can_output",
        "can_interface": "can0",
        "can_id": "0x400",
        "data_offset": 0,
        "data_length": 1,
        "scale_factor": 2.55
    },
    "zones": [
        {
            "name": "low",
            "min_lux": 0.0,
            "max_lux": 10.0,
            "curve_type": "linear",
            "min_brightness": 10,
            "max_brightness": 25,
            "up_gain": 0.20,
            "down_gain": 0.08
        },
        {
            "name": "medium",
            "min_lux": 10.0,
            "max_lux": 1000.0,
            "curve_type": "logarithmic",
            "min_brightness": 25,
            "max_brightness": 70,
            "up_gain": 0.22,
            "down_gain": 0.10
        },
        {
            "name": "high",
            "min_lux": 1000.0,
            "max_lux": 50000.0,
            "curve_type": "logarithmic",
            "min_brightness": 70,
            "max_brightness": 95,
            "up_gain": 0.30,
            "down_gain": 0.15
        }
    ],
    "control": {
        "loop_period_ms": 100,
        "error_threshold": 1.0,
        "min_step_size": 1.0,
        "hysteresis_percent": 12.0,
        "manual_timeout_seconds": 30
    },
    "control_interface": {
        "tcp": {
            "enabled": true,
            "port": 9000,
            "bind_address": "127.0.0.1"
        },
        "unix": {
            "enabled": true,
            "socket_path": "/var/run/als-dimmer.sock"
        }
    }
}
\end{lstlisting}

\subsection{Appendix A.3: Development/Testing Configuration}

\begin{lstlisting}[caption={File-Based Test Configuration (Full)}, basicstyle=\ttfamily\scriptsize]
{
    "sensor": {
        "type": "file",
        "file_path": "/tmp/test_lux.txt"
    },
    "output": {
        "type": "file",
        "file_path": "/tmp/test_brightness.txt"
    },
    "zones": [
        {
            "name": "test_zone_low",
            "min_lux": 0.0,
            "max_lux": 100.0,
            "curve_type": "linear",
            "min_brightness": 10,
            "max_brightness": 40,
            "up_gain": 0.30,
            "down_gain": 0.15
        },
        {
            "name": "test_zone_high",
            "min_lux": 100.0,
            "max_lux": 10000.0,
            "curve_type": "logarithmic",
            "min_brightness": 40,
            "max_brightness": 90,
            "up_gain": 0.30,
            "down_gain": 0.15
        }
    ],
    "control": {
        "loop_period_ms": 50,
        "error_threshold": 0.1,
        "min_step_size": 0.5,
        "hysteresis_percent": 5.0,
        "manual_timeout_seconds": 10
    },
    "control_interface": {
        "tcp": {
            "enabled": true,
            "port": 9000,
            "bind_address": "127.0.0.1"
        },
        "unix": {
            "enabled": false
        }
    }
}
\end{lstlisting}

\clearpage

\section{Build and Installation Reference}

\subsection{Appendix B.1: Dependencies}

\subsubsection{Ubuntu/Debian}

\begin{lstlisting}[language=bash, caption={Install Build Dependencies (Ubuntu 20.04+)}]
sudo apt-get update
sudo apt-get install -y \
    build-essential \
    cmake \
    git \
    libi2c-dev \
    nlohmann-json3-dev \
    libspdlog-dev \
    libsocketcan-dev
\end{lstlisting}

\subsubsection{Fedora/RHEL}

\begin{lstlisting}[language=bash, caption={Install Build Dependencies (Fedora 34+)}]
sudo dnf install -y \
    gcc-c++ \
    cmake \
    git \
    i2c-tools-devel \
    json-devel \
    spdlog-devel \
    libsocketcan-devel
\end{lstlisting}

\subsection{Appendix B.2: Complete Build Sequence}

\begin{lstlisting}[language=bash, caption={Full Build and Install}]
# Clone repository
git clone https://github.com/your-org/als-dimmer.git
cd als-dimmer

# Create build directory
mkdir build && cd build

# Configure with all options
cmake -DCMAKE_BUILD_TYPE=Release \
      -DENABLE_TESTING=ON \
      -DENABLE_OPT4001=ON \
      -DENABLE_CAN=ON \
      -DENABLE_DDCCI=ON \
      -DINSTALL_SYSTEMD=ON \
      ..

# Build with all cores
make -j$(nproc)

# Run tests
make test

# Install (requires root)
sudo make install

# Enable and start service
sudo systemctl enable als-dimmer
sudo systemctl start als-dimmer
\end{lstlisting}

\subsection{Appendix B.3: Yocto Integration}

\begin{lstlisting}[caption={als-dimmer.bb Yocto Recipe}, basicstyle=\ttfamily\scriptsize]
SUMMARY = "ALS-Dimmer Adaptive Brightness Control Daemon"
LICENSE = "MIT"
LIC_FILES_CHKSUM = "file://LICENSE;md5=..."

SRC_URI = "git://github.com/your-org/als-dimmer.git;protocol=https"
SRCREV = "${AUTOREV}"

S = "${WORKDIR}/git"

DEPENDS = "nlohmann-json spdlog"

inherit cmake systemd

SYSTEMD_SERVICE_${PN} = "als-dimmer.service"
SYSTEMD_AUTO_ENABLE = "enable"

EXTRA_OECMAKE = "-DCMAKE_BUILD_TYPE=Release \
                 -DINSTALL_SYSTEMD=ON"

do_install_append() {
    install -d ${D}${sysconfdir}/als-dimmer
    install -m 0644 ${WORKDIR}/automotive-config.json \
        ${D}${sysconfdir}/als-dimmer/config.json
}

FILES_${PN} += "${systemd_system_unitdir}/als-dimmer.service \
                ${sysconfdir}/als-dimmer/config.json"
\end{lstlisting}

\clearpage

\section{Mathematical Details}

\subsection{Appendix C.1: Detailed Curve Formulas}

\subsubsection{Linear Curve (General Form)}

\begin{equation}
B(L) = B_{min} + m \cdot (L - L_{min})
\end{equation}

Where slope $m$ is:
\begin{equation}
m = \frac{B_{max} - B_{min}}{L_{max} - L_{min}}
\end{equation}

\textbf{Example with numbers:} Zone from 0--5 lux mapping to 5--20\% brightness:
\begin{align*}
m &= \frac{20 - 5}{5 - 0} = 3.0 \\
B(L) &= 5 + 3.0 \cdot (L - 0) = 5 + 3L
\end{align*}

At $L = 2$ lux: $B(2) = 5 + 3 \times 2 = 11\%$

\subsubsection{Logarithmic Curve (Base-10)}

\begin{equation}
B(L) = B_{min} + (B_{max} - B_{min}) \cdot \frac{\log_{10}(L / L_{min})}{\log_{10}(L_{max} / L_{min})}
\end{equation}

\textbf{Example:} Zone from 5--500 lux mapping to 20--60\% brightness:
\begin{align*}
B(L) &= 20 + 40 \cdot \frac{\log_{10}(L / 5)}{\log_{10}(500 / 5)} \\
     &= 20 + 40 \cdot \frac{\log_{10}(L / 5)}{2.0}
\end{align*}

At $L = 50$ lux:
\begin{align*}
B(50) &= 20 + 40 \cdot \frac{\log_{10}(50 / 5)}{2.0} \\
      &= 20 + 40 \cdot \frac{1.0}{2.0} = 40\%
\end{align*}

\subsection{Appendix C.2: Step Size Examples}

Given current brightness $B_c = 30\%$, target $B_t = 60\%$, error $e = 30\%$:

\textbf{Brightening ($k_{up} = 0.25$):}
\begin{align*}
\Delta B &= 0.25 \times 30 = 7.5\% \\
B_{new} &= 30 + 7.5 = 37.5\%
\end{align*}

\textbf{Dimming ($k_{down} = 0.10$):}
\begin{align*}
\Delta B &= 0.10 \times 30 = 3.0\% \\
B_{new} &= 60 - 3.0 = 57.0\%
\end{align*}

\textbf{Convergence time estimation:}
For error $e_0$ and gain $k$, after $n$ iterations:
\begin{equation}
e_n = e_0 \cdot (1 - k)^n
\end{equation}

To reach $e_n < 1\%$ from $e_0 = 30\%$ with $k = 0.25$:
\begin{align*}
1 &> 30 \cdot (0.75)^n \\
n &> \frac{\log(1/30)}{\log(0.75)} \approx 11.8 \text{ iterations}
\end{align*}

At 100 ms per iteration: $\approx 1.2$ seconds to converge.

\clearpage

\section{References and Further Reading}

\subsection{Appendix D.1: Project Resources}

\begin{itemize}[leftmargin=*]
    \item \textbf{Repository:} \texttt{https://github.com/your-org/als-dimmer}
    \item \textbf{Issue Tracker:} \texttt{https://github.com/your-org/als-dimmer/issues}
    \item \textbf{Documentation:} \texttt{https://als-dimmer.readthedocs.io}
    \item \textbf{Author Contact:} albert.david@gmail.com
\end{itemize}

\subsection{Appendix D.2: Hardware Datasheets}

\begin{itemize}[leftmargin=*]
    \item \textbf{TI OPT4001:} Texas Instruments, ``OPT4001 High-Speed High-Precision Light-to-Digital Sensor,'' Datasheet, 2022.
    \item \textbf{DDC/CI Specification:} VESA, ``Display Data Channel Command Interface (DDC/CI) Standard,'' Version 1.1, 2004.
    \item \textbf{I2C Specification:} NXP Semiconductors, ``I2C-bus specification and user manual,'' Rev. 7.0, 2021.
\end{itemize}

\subsection{Appendix D.3: Standards and Protocols}

\begin{itemize}[leftmargin=*]
    \item \textbf{CAN Bus:} ISO 11898-1:2015, ``Road vehicles -- Controller area network (CAN)''
    \item \textbf{DDC/CI:} VESA MCCS (Monitor Control Command Set) Standard
    \item \textbf{Systemd:} \texttt{https://www.freedesktop.org/wiki/Software/systemd/}
\end{itemize}

\subsection{Appendix D.4: Scientific Background}

\begin{itemize}[leftmargin=*]
    \item Weber, E. H. (1834). \textit{De Pulsu, Resorptione, Auditu et Tactu.} Leipzig: Koehler.
    \item Fechner, G. T. (1860). \textit{Elemente der Psychophysik.} Leipzig: Breitkopf und Härtel.
    \item Hood, D. C., \& Finkelstein, M. A. (1986). ``Sensitivity to Light.'' In \textit{Handbook of Perception and Human Performance}, Vol. 1, K. R. Boff, L. Kaufman, \& J. P. Thomas (Eds.), New York: Wiley.
    \item Fairchild, M. D. (2013). \textit{Color Appearance Models}, 3rd Ed. Wiley-IS\&T Series in Imaging Science and Technology.
\end{itemize}

\subsection{Appendix D.5: Related Open-Source Projects}

\begin{itemize}[leftmargin=*]
    \item \textbf{Calise:} Camera-based adaptive brightness for Linux desktops
    \item \textbf{brightnessctl:} Simple brightness control utility for Linux
    \item \textbf{ddcutil:} DDC/CI communication tool for monitors
    \item \textbf{iio-sensor-proxy:} IIO sensor interface daemon (includes ALS support)
\end{itemize}

\subsection{Appendix D.6: Suggested Reading for Implementation}

For teams implementing adaptive brightness systems:

\begin{enumerate}[leftmargin=*]
    \item \textbf{Fundamentals:} Fairchild's \textit{Color Appearance Models} (Chapter 2: Visual Response) provides excellent background on human vision.
    \item \textbf{Automotive context:} SAE J1757 ``Recommended Practice for Selection, Installation, and Testing of Automatic Headlamp Beam Switching Devices'' offers insights into automotive ALS applications.
    \item \textbf{Control theory:} Åström \& Murray's \textit{Feedback Systems: An Introduction for Scientists and Engineers} covers proportional control and error-based systems.
    \item \textbf{Embedded Linux:} Yaghmour's \textit{Building Embedded Linux Systems} for deployment strategies and systemd integration.
\end{enumerate}

\vspace{2em}
\noindent\rule{\textwidth}{0.4pt}

\begin{center}
\textbf{End of Document}
\end{center}
