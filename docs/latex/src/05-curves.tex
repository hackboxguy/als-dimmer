%------------------------
% Section 5: Brightness Curves
%------------------------
\section{Brightness Curve Mathematics}

Once a zone is selected, the system must map the current lux value to a target brightness percentage. This section presents the two primary curve types used in adaptive brightness control: \textbf{linear} and \textbf{logarithmic}.

\subsection{Linear Curves}

Linear curves provide a straightforward proportional relationship between ambient light and display brightness.

\subsubsection{Formula}

\begin{equation}
B(L) = B_{min} + \frac{B_{max} - B_{min}}{L_{max} - L_{min}} \cdot (L - L_{min})
\label{eq:linear_curve}
\end{equation}

Where:
\begin{itemize}[leftmargin=*]
    \item $B(L)$ = Target brightness (\%) for ambient lux $L$
    \item $B_{min}$ = Minimum brightness for the zone (\%)
    \item $B_{max}$ = Maximum brightness for the zone (\%)
    \item $L_{min}$ = Lower lux boundary of the zone
    \item $L_{max}$ = Upper lux boundary of the zone
    \item $L$ = Current ambient lux reading
\end{itemize}

\subsubsection{Example Calculation}

Consider a \textbf{night zone} with the following parameters:
\begin{itemize}[leftmargin=*]
    \item Lux range: 0--5 lux
    \item Brightness range: 5--20\%
    \item Current ambient: 2.5 lux
\end{itemize}

Applying Equation~\ref{eq:linear_curve}:

\begin{align*}
B(2.5) &= 5 + \frac{20 - 5}{5 - 0} \cdot (2.5 - 0) \\
       &= 5 + \frac{15}{5} \cdot 2.5 \\
       &= 5 + 3 \cdot 2.5 \\
       &= 5 + 7.5 \\
       &= 12.5\%
\end{align*}

\textbf{Result:} At 2.5 lux, the display brightness is set to 12.5\%.

\subsubsection{When to Use Linear Curves}

Linear curves work well when:
\begin{itemize}[leftmargin=*]
    \item The lux range is narrow (1--2 orders of magnitude)
    \item Fine-grained control is needed (e.g., night zones)
    \item Human perception is approximately linear within the range
    \item Predictable, proportional adjustments are desired
\end{itemize}

\subsection{Logarithmic Curves}

Logarithmic curves align with the Weber-Fechner law, providing perceptually uniform adjustments across wide dynamic ranges.

\subsubsection{Formula}

\begin{equation}
B(L) = B_{min} + (B_{max} - B_{min}) \cdot \frac{\log(1 + (L - L_{min}))}{\log(1 + (L_{max} - L_{min}))}
\label{eq:log_curve}
\end{equation}

Where all variables are defined as in Equation~\ref{eq:linear_curve}, and $\log$ denotes the natural logarithm ($\ln$) or base-10 logarithm (both work; base-10 is more intuitive).

\textbf{Implementation Note:} The formula uses $\log(1 + x)$ instead of $\log(L / L_{min})$ for improved numerical stability and to avoid potential edge cases when $L$ approaches $L_{min}$. This shifted logarithm approach maintains the perceptual uniformity property while being more robust in practice.

\subsubsection{Example Calculation}

Consider an \textbf{indoor zone} with:
\begin{itemize}[leftmargin=*]
    \item Lux range: 5--500 lux
    \item Brightness range: 20--60\%
    \item Current ambient: 50 lux
\end{itemize}

Using Equation~\ref{eq:log_curve} with base-10 logarithm:

\begin{align*}
B(50) &= 20 + (60 - 20) \cdot \frac{\log_{10}(1 + (50 - 5))}{\log_{10}(1 + (500 - 5))} \\
      &= 20 + 40 \cdot \frac{\log_{10}(1 + 45)}{\log_{10}(1 + 495)} \\
      &= 20 + 40 \cdot \frac{\log_{10}(46)}{\log_{10}(496)} \\
      &= 20 + 40 \cdot \frac{1.663}{2.696} \\
      &= 20 + 40 \cdot 0.617 \\
      &= 20 + 24.7 \\
      &= 44.7\%
\end{align*}

\textbf{Result:} At 50 lux, brightness is 44.7\%. Note that with the shifted logarithm, 50 lux produces slightly higher brightness than the mathematical midpoint, providing better visibility in the lower-to-mid lux range.

\subsubsection{When to Use Logarithmic Curves}

Logarithmic curves are ideal when:
\begin{itemize}[leftmargin=*]
    \item The lux range spans 2+ orders of magnitude
    \item Perceptual uniformity is critical
    \item The zone covers diverse lighting conditions (e.g., indoor, outdoor)
    \item Alignment with Weber-Fechner law is desired
\end{itemize}

\subsection{Curve Comparison}

Figure~\ref{fig:curve_comparison} illustrates the difference between linear and logarithmic curves for the same lux range (5--500 lux, targeting 20--60\% brightness).

\begin{figure}[ht]
\centering
\begin{tikzpicture}[scale=1.0]
    % Axes
    \draw[->] (0,0) -- (10,0) node[right] {Ambient Light (lux)};
    \draw[->] (0,0) -- (0,5) node[above] {Brightness (\%)};

    % Axis labels
    \foreach \x/\label in {0/5, 2.5/62.5, 5/125, 7.5/250, 10/500} {
        \draw (\x,0.1) -- (\x,-0.1) node[below, font=\footnotesize] {\label};
    }
    \foreach \y/\label in {0/20, 1.25/30, 2.5/40, 3.75/50, 5/60} {
        \draw (0.1,\y) -- (-0.1,\y) node[left, font=\footnotesize] {\label};
    }

    % Linear curve (green)
    \draw[indoorzone, very thick] (0,0) -- (10,5) node[pos=0.9, below, indoorzone] {Linear};

    % Logarithmic curve (blue)
    \draw[nightzone, very thick, domain=0:10, samples=100]
        plot (\x, {5 * ln(1 + \x/0.5) / ln(1 + 10/0.5)});

    % Logarithmic label (placed at middle of curve)
    \node[nightzone, above] at (5, 4.2) {Logarithmic};

    % Grid
    \draw[gray!65, dashed] (0,0) grid (10,5);
\end{tikzpicture}
\caption{Linear vs. Logarithmic Curves (5--500 lux $\rightarrow$ 20--60\%)}
\label{fig:curve_comparison}
\end{figure}

\textbf{Observation:} The logarithmic curve provides more sensitivity at lower lux values (steeper slope), which aligns better with human perception in environments transitioning from dim to moderate light.

\subsection{Connection to Weber-Fechner Law}

Recall from Section~3.3 that the Weber-Fechner law states:

\begin{equation}
P \propto \log(S)
\label{eq:weber_fechner_recall}
\end{equation}

Where $P$ is perceived intensity and $S$ is stimulus intensity.

Logarithmic brightness curves directly implement this relationship:
\begin{itemize}[leftmargin=*]
    \item \textbf{Stimulus:} Ambient light (lux)
    \item \textbf{Perceived change:} Display brightness adjustment
    \item \textbf{Result:} Equal percentage changes in lux produce equal perceived brightness changes
\end{itemize}

This is why logarithmic curves feel "natural" to users across wide dynamic ranges, while linear curves can feel too aggressive in bright conditions or too sluggish in dim ones.

\subsection{Practical Curve Selection Guidelines}

\begin{table}[h]
\centering
\caption{Recommended Curve Types by Zone}
\label{tab:curve_selection}
\begin{tabular}{@{}llll@{}}
\toprule
\textbf{Zone} & \textbf{Lux Range} & \textbf{Dynamic Range} & \textbf{Recommended Curve} \\
\midrule
Night         & 0--5 lux          & 1 order (narrow)     & Linear \\
Indoor        & 5--500 lux        & 2 orders (wide)      & Logarithmic \\
Outdoor       & 500--10,000 lux   & 2 orders (wide)      & Logarithmic \\
Extreme       & 10,000+ lux       & 1 order (narrow)     & Linear \\
\bottomrule
\end{tabular}
\end{table}

\subsection{Advanced: Hybrid and Custom Curves}

Some implementations use hybrid approaches:

\begin{itemize}[leftmargin=*]
    \item \textbf{Piecewise curves:} Linear in some sub-ranges, logarithmic in others
    \item \textbf{Exponential curves:} $B(L) = B_{min} \cdot (L / L_{min})^k$ where $k < 1$ for sub-linear growth
    \item \textbf{User-tunable curves:} Allow end users to bias toward dimmer/brighter via a single parameter
    \item \textbf{Learned curves:} Machine learning models that adapt to individual user preferences over time
\end{itemize}

However, the vast majority of practical systems achieve excellent results with the simple linear and logarithmic formulas presented here.

\begin{quote}
\textbf{Hardware transfer assumption:} All equations in this section assume the display exposes a roughly linear 0--100\% control (or that the backlight driver already compensates for its own non-linearity). If a panel's PWM-to-luminance curve is strongly non-linear, add a calibration layer or lookup table so that ``percent'' commands correspond to actual nits; otherwise the lux $\rightarrow$ percent curves here will not yield consistent perceptual changes.
\end{quote}

\subsection{Summary}

Brightness curves translate lux measurements into display brightness percentages. The choice between linear and logarithmic depends on the zone's dynamic range:

\begin{itemize}[leftmargin=*]
    \item \textbf{Narrow ranges (night, extreme):} Linear curves provide intuitive, proportional control
    \item \textbf{Wide ranges (indoor, outdoor):} Logarithmic curves align with human perception
    \item \textbf{Mathematical foundation:} Logarithmic curves implement the Weber-Fechner law
\end{itemize}

The next section addresses how quickly these brightness changes should be applied.
