%------------------------
% Section 11: Control Interface
%------------------------
\section{Runtime Control Interface}

ALS-Dimmer provides a JSON-based control interface over TCP and Unix domain sockets, enabling runtime interaction from user applications, HMI frameworks, and command-line tools.

\subsection{Socket Configuration}

The daemon can be configured to listen on multiple socket types:

\begin{lstlisting}[caption={Socket Configuration}]
"control_interface": {
    "tcp": {
        "enabled": true,
        "port": 9000,
        "bind_address": "127.0.0.1"
    },
    "unix": {
        "enabled": true,
        "socket_path": "/tmp/als-dimmer.sock"
    }
}
\end{lstlisting}

\textbf{Typical usage patterns:}
\begin{itemize}[leftmargin=*]
    \item \textbf{TCP socket:} Development, testing, remote access
    \item \textbf{Unix domain socket:} Production IPC with Android/QNX HMI frameworks (lower latency, better security)
\end{itemize}

\subsection{Protocol: JSON v1.0}

All commands and responses use JSON format.

\subsubsection{Command Structure}

\begin{lstlisting}[caption={Command Format}]
{
    "version": "1.0",
    "command": "COMMAND_NAME",
    "params": { /* Optional command-specific parameters */ }
}
\end{lstlisting}

\subsubsection{Response Structure}

\begin{lstlisting}[caption={Response Format}]
{
    "version": "1.0",
    "status": "success" | "error",
    "data": { /* Response data */ },
    "error": "Error message if status is error"
}
\end{lstlisting}

\subsection{Available Commands}

\subsubsection{1. GET\_STATUS}

Query current daemon status, brightness, mode, and sensor reading.

\textbf{Request:}
\begin{lstlisting}[]
{
    "version": "1.0",
    "command": "get_status"
}
\end{lstlisting}

\textbf{Response:}
\begin{lstlisting}[]
{
    "version": "1.0",
    "status": "success",
    "data": {
        "mode": "auto",
        "current_brightness": 45,
        "target_brightness": 50,
        "current_lux": 120.5,
        "current_zone": "indoor",
        "sensor_operational": true,
        "output_operational": true
    }
}
\end{lstlisting}

\subsubsection{2. SET\_MODE}

Change operating mode (AUTO, MANUAL). The transient \texttt{MANUAL\_TEMPORARY} state is managed by the daemon and cannot be requested directly; attempts to set it are rejected with an \texttt{INVALID\_PARAMS} error.

\textbf{Request:}
\begin{lstlisting}[]
{
    "version": "1.0",
    "command": "set_mode",
    "params": {
        "mode": "manual"
    }
}
\end{lstlisting}

\textbf{Valid modes:} \texttt{"auto"}, \texttt{"manual"} (requests for \texttt{"manual\_temporary"} are rejected)

\textbf{Response:}
\begin{lstlisting}[]
{
    "version": "1.0",
    "status": "success",
    "data": {
        "mode": "manual"
    }
}
\end{lstlisting}

\subsubsection{3. SET\_BRIGHTNESS}

Set brightness to specific value. If the control loop is currently in \texttt{AUTO}, this command triggers the read-only \texttt{MANUAL\_TEMPORARY} state (auto-resume timer starts); when already in \texttt{MANUAL}, the mode stays manual.

\textbf{Request:}
\begin{lstlisting}[]
{
    "version": "1.0",
    "command": "set_brightness",
    "params": {
        "brightness": 75
    }
}
\end{lstlisting}

\textbf{Valid range:} 0--100 (integer)

\textbf{Response:}
\begin{lstlisting}[]
{
    "version": "1.0",
    "status": "success",
    "data": {
        "brightness": 75,
        "mode": "manual_temporary"
    }
}
\end{lstlisting}

\subsubsection{4. ADJUST\_BRIGHTNESS}

Adjust brightness by delta. Like \texttt{set\_brightness}, invoking this while in \texttt{AUTO} transitions the daemon into \texttt{MANUAL\_TEMPORARY}; if already in \texttt{MANUAL}, the mode does not change.

\textbf{Request:}
\begin{lstlisting}[]
{
    "version": "1.0",
    "command": "adjust_brightness",
    "params": {
        "delta": 10
    }
}
\end{lstlisting}

\textbf{Delta range:} -100 to +100 (integer). Result is clamped to [0, 100].

\textbf{Response:}
\begin{lstlisting}[]
{
    "version": "1.0",
    "status": "success",
    "data": {
        "brightness": 55,
        "mode": "manual_temporary"
    }
}
\end{lstlisting}

\textbf{Use case:} User presses brightness up/down buttons on steering wheel or display. System adjusts brightness but reverts to AUTO after timeout; clients do not (and cannot) explicitly request \texttt{MANUAL\_TEMPORARY}.

\subsubsection{5. GET\_CONFIG}

Retrieve current configuration (zones, control parameters).

\textbf{Request:}
\begin{lstlisting}[]
{
    "version": "1.0",
    "command": "get_config"
}
\end{lstlisting}

\textbf{Response:}
\begin{lstlisting}[basicstyle=\ttfamily\scriptsize]
{
    "version": "1.0",
    "status": "success",
    "data": {
        "zones": [
            {
                "name": "night",
                "min_lux": 0.0,
                "max_lux": 5.0,
                "curve_type": "linear",
                "min_brightness": 5,
                "max_brightness": 20,
                "up_gain": 0.25,
                "down_gain": 0.10
            },
            // ... additional zones
        ],
        "control": {
            "loop_period_ms": 150,
            "error_threshold": 0.5,
            "min_step_size": 1.0,
            "hysteresis_percent": 10.0,
            "manual_timeout_seconds": 45
        }
    }
}
\end{lstlisting}

\subsection{Error Responses}

If a command fails, the daemon returns an error response:

\begin{lstlisting}[]
{
    "version": "1.0",
    "status": "error",
    "error": "Invalid brightness value: 150 (must be 0-100)"
}
\end{lstlisting}

\textbf{Common error conditions:}
\begin{itemize}[leftmargin=*]
    \item Invalid JSON syntax
    \item Unknown command
    \item Missing required parameters
    \item Out-of-range values
    \item Internal daemon error (sensor/output failure)
\end{itemize}

\subsection{Command-Line Examples}

\subsubsection{Using netcat (nc) with TCP socket}

\begin{lstlisting}[language=bash, caption={Query Status via TCP}]
echo '{"version":"1.0","command":"get_status"}' | \
    nc 127.0.0.1 9000
\end{lstlisting}

\begin{lstlisting}[language=bash, caption={Set Brightness to 80\%}]
echo '{"version":"1.0","command":"set_brightness",
      "params":{"brightness":80}}' | \
    nc 127.0.0.1 9000
\end{lstlisting}

\subsubsection{Using netcat with Unix socket}

\begin{lstlisting}[language=bash, caption={Query Status via Unix Socket}]
echo '{"version":"1.0","command":"get_status"}' | \
    nc -U /tmp/als-dimmer.sock
\end{lstlisting}

\subsubsection{Using curl (if HTTP wrapper is available)}

Some deployments wrap the socket interface with an HTTP server:

\begin{lstlisting}[language=bash, caption={HTTP REST API (Custom Wrapper)}]
curl -X POST http://localhost:8080/als-dimmer/status

curl -X POST http://localhost:8080/als-dimmer/brightness \
    -d '{"brightness": 60}'
\end{lstlisting}

\subsection{Integration with HMI Frameworks}

\subsubsection{Android IVI Integration}

Android applications can communicate via Unix socket using Java/Kotlin:

\begin{lstlisting}[, caption={Android Integration (Java)}, basicstyle=\ttfamily\scriptsize]
import android.net.LocalSocket;
import android.net.LocalSocketAddress;

// Connect to daemon
LocalSocket socket = new LocalSocket();
socket.connect(new LocalSocketAddress(
    "/tmp/als-dimmer.sock",
    LocalSocketAddress.Namespace.FILESYSTEM));

// Send command
String command = "{\"version\":\"1.0\"," +
                 "\"command\":\"get_status\"}";
socket.getOutputStream().write(command.getBytes());

// Read response
byte[] buffer = new byte[4096];
int len = socket.getInputStream().read(buffer);
String response = new String(buffer, 0, len);

// Parse JSON response
JSONObject json = new JSONObject(response);
int brightness = json.getJSONObject("data")
                     .getInt("current_brightness");
\end{lstlisting}

\subsubsection{QNX Integration}

QNX applications use similar Unix socket approach with POSIX APIs.

\subsection{Security Considerations}

\begin{itemize}[leftmargin=*]
    \item \textbf{Unix socket permissions:} Typically 0660 (owner + group only)
    \item \textbf{TCP bind address:} Bind to 127.0.0.1 (localhost only) in production
    \item \textbf{Authentication:} Future enhancement (v2.0 protocol may add token-based auth)
    \item \textbf{Rate limiting:} Daemon rejects excessive command rates to prevent abuse
\end{itemize}

\subsection{Protocol Versioning}

The \texttt{version} field enables protocol evolution:

\begin{itemize}[leftmargin=*]
    \item \textbf{v1.0:} Current protocol as described
    \item \textbf{Future versions:} May add new commands, deprecate old ones
    \item \textbf{Backward compatibility:} Daemon should support older protocol versions when feasible
\end{itemize}

\subsection{Summary}

The control interface provides:

\begin{itemize}[leftmargin=*]
    \item \textbf{Runtime control:} Mode changes, brightness adjustments, status queries
    \item \textbf{Dual socket support:} TCP (development) and Unix (production)
    \item \textbf{JSON protocol:} Human-readable, easy to integrate
    \item \textbf{HMI integration:} Android, QNX, web UIs can control daemon
    \item \textbf{Command-line access:} Simple testing with netcat
\end{itemize}

\subsection{Source Code Availability}

The complete source code for the ALS-Dimmer system is available on github at:

\url{https://github.com/hackboxguy/als-dimmer.git}


%The next section covers deployment: building, installing, and running ALS-Dimmer in production.
